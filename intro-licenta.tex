\documentclass[a4paper]{article}
\usepackage{amsmath}
\usepackage{graphicx}
\usepackage{hyperref}
\usepackage[english]{babel}
\usepackage[utf8]{inputenc}

\title{Introducere}
\author{Maxim Dragoș Andrei}
\date{}

\begin{document}
\maketitle

\section{Introducere}
\subsection{Context}
\begin{normalsize}

\hspace{5mm}Posibilitatea călătoriei în timp, alături de posibilitatea teleportării reprezintă un subiect prezent atât în operele științifico-fantastice cât și în cercetările actuale. \\

\hspace{5mm}În ultimul secol s-au înregistrat progrese semnificative referitoare la transportul fotonilor pe distanțe cât mai mari, dar și în cât mai buna înțelegere a geometriei spațiu-timp.
\end{normalsize}

\subsection{Motivație}
\begin{normalsize}
\hspace{5mm}Scopul lucrării este de a surprinde importanța modelării matematice în investigarea problemei calatoriei în timp și a teleportării. Pe parcursul lucrării vor fi prezentate modele utilizate în cadrul experiențelor ce au avut loc de-a lungul timpului.\\ 

\hspace{5mm}Prima parte va fi constituită din o scurtă prezentare a teoriei necesare întelegerii modelelor matematice. În continuare, pentru a evidenția factorii implicați în problema aflată în discuție, se vor analiza 3 modele. În cele din urmă va fi prezentat impactul avut de informatică asupra domeniului.
\end{normalsize}
\end{document}
